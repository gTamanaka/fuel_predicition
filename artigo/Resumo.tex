\begin{resumoumacoluna}
O presente trabalho visa entender se o preço do combustível pode ser previsto utilizando os 
valores disponibilizados pela Agência Nacional do Petróleo, Gás Natural e Biocombustível para analisar a viabilidade da previsão de preços utilizando aprendizado de máquina. Para tanto, este artigo analisa os dados do preço da gasolina comum para o estado de São Paulo.
 
 O artigo faz uso de quatro estratégias diferentes de regressão, amplamente utilizadas para esse fim de previsão, para os testes. Finalmente, os valores previstos são comparados com o esperado e analisados as viabilidades dos modelos.

 \vspace{\onelineskip}
 
 \noindent
 \textbf{Palavras-chave}: aprendizado de máquinas, previsão, \textit{machine learning forecasting},regressão linear, \textit{state vector regression}.
\end{resumoumacoluna}
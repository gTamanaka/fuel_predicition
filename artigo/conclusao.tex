Por meio da análise dos resultados obtidos é notável que os modelos de regressão não são capazes de predizer
os valores dos combustíveis. Muito se deve a natureza de como o preço do combustível é definido, dependo
muito menos da produção e muito mais das interferências políticas, portanto, seria necessário uma análise 
de outro fatores como por exemplo o valor do combustível internacionamentel e das decissões tomadas 
pelo o Estado durante o período, este último dificilmente aplicável a um modelo de aprendizado de máquina.

Também é notável que os dados poderiam possuir mais homgeneidade para suas coletas, como por exemplo o mesmo 
número de postos por estado. 

Por tudo isso, fica a trabalhos futuros adicionar estas, já citadas, camadas de complexidade à análise, desta
forma sendo capaz de cobrir outras propriedades do projeto.
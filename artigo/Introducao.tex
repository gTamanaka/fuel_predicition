Uma das mais notáveis aplicações do aprendizado de máquina é a obtenção de modelos que possam prever 
o comportamento de uma série temporal, como por exemplo o o preço de ações e o consumo de eletricidade \cite{Eletricity}.
A determinação do comportamento de uma série temporal permite a tomada de decisões a priori da ocorrência de um cenário,
isto representa uma vantagem comercial competitiva em um cenário cada vez mais otimizado a resultados.  Análise do preço do combustível, proposto neste trabalho, pode permitir a empresas de logísticas determinarem preços
de entregas, valores de fretez cobrados e otimização do custo operacional. Consumidores também se beneficiariam
ao poderem se adiantarem a variação do preço e decidirem pelo melhor momento de realizar uma compra ou não. 

Modelos de regressão são utéis na tentativa de prever o comportamento de uma serie temporal, 
O presente trabalho se propõem em analisar a viabilidade da utilização de  modelos aprendizado de máquina para a previsão dos preços por meio regressão, bem como comparar os resultados obtidos por diferentes modelos para mesma previsão a fim de entender os quais produzem melhores aproximação para o valor da gasolina no estado de São Paulo.

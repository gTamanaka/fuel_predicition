Por meio da análise dos resultados obtidos é notável que os modelos de regressão não são capazes de predizer
os valores dos combustíveis. Provavelmente devido a natureza de como o preço do combustível é definido, dependo
muito menos da produção e muito mais das interferências políticas, portanto, seria necessário uma análise 
de outro fatores como por exemplo o valor do combustível internacionalmente e das decisões tomadas 
pelo o Estado durante o período, este último dificilmente aplicável a um modelo de aprendizado de máquina, além de questões envolvendo a extração do petróleo.

Também é notável que os dados possuem muito ruído, é difícil saber se é apenas ruído devido a uma coleta ruim ou se 
é um padrão esperado quando medidas governamentais são tomadas para amenizar o preço ou por interesses do mercado. Além do mais, a homogeneidade dos dados é questionável em suas coletas, como por exemplo o número de postos por estado não é o mesmo, isto torna os dados desafiador julgar a qualidade dos dados.

Fica para trabalhos futuros adicionar as já citadas características políticas do preço, estas adicionariam camadas de complexidade à análise desta forma sendo capaz de cobrir a natureza do ruído do \textit{dataset}.


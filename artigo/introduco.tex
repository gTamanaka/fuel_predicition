% Consiste na apresentação do assunto, dos objetivos e demais elementos
% necessários para se ter uma visão de conjunto do tema. Para tanto deve:
% a) especificar qual é o assunto, objeto de estudo;
% b) esclarecer sobre que ponto de vista o assunto será abordado; e
% c) apresentar as justificativas que levaram o autor a escolher o tema, o
% problema de pesquisa, a hipótese de estudo, o objetivo pretendido e as
% razões de escolha do método.
%  A introdução tem a função de despertar o interesse do leitor em relação
% ao texto. Assim, recomenda-se que a introdução seja a última parte do trabalho a ser
% redigida.

% *
%  Doutor em Administração. Professor nos Cursos de Graduação e Pós-graduação da FSG. Endereço
% eletrônico: xxx.xxxxx@fsg.br. 
% Obs. O desenvolvimento do artigo científico pode ser organizado em
% uma única parte (2 DESENVOLVIMENTO) ou dividido em seções e subseções (2
% FUNDAMENTAÇÃO TEÓRICA, 3 METODOLOGIA E 4 RESULTADOS). Os
% termos desenvolvimento, fundamentação teórica, metodologia e resultados podem ser
% subcapítulos por títulos que sejam mais representativos e significativos ao texto. 



% Este item descreve a delimitação do universo estudado (população e
% amostra), o método e as técnicas de coleta de dados, como foram desenvolvidas as
% etapas da pesquisa e suas limitações. Deve sempre ser escrito com o verbo no tempo
% passado, pois descreve o que já foi investigado.
% Nesta parte do trabalho podem ser usados subtítulos para as partes. 

O \textit{dataset} utilizado é disponiblizado pela Agência Nacional do Petróleo, Gás Natural e Biocombustível.
Os dados são a série histórica semanal dos preços e margens de venda de 2004 até abril de 2019. Os dados
podem ser consumidos para o Brasil inteiro, por regiões, por estados ou por municípios.

Os dados disponvéis para uso são a data incial, data final, região, estado,
produto, número de postos pesquisados, unidade de medida, preço médio da revenda,
desvio padrão da revenda, preço mínimo da revenda, preço máximo da revenda,
margem média revenda, coeficiente de variação da revenda, preço médio distribuição,
desvio médio distribuição, preço mínimo da distribuição, preço máximo da distribuição,
margem média distribuição, coeficiente de variação da distribuição, ano e mês.
O presente trabalho visa apenas prever o valor final que chega ao consumidor, por tanto,
apenas os valores de data e preços médio foram utilizados. A análise tabém
limitou-se apenas ao estado de São Paulo e a gasolina, visto que o objetivo
é validar a possibilidade de utilizar um método de aprendizado de máquina para a prever
o preço.
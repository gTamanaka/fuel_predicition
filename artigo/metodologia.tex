% É a parte erudita do texto. A importância deste item refere-se à
% necessidade do leitor em saber o que existe na literatura correlata, as informações e
% sugestões sobre o problema em estudo. Ou seja, são os fatores existentes no estoque de
% conhecimento e que são adequados ao problema.
% Para elaborar um referencial teórico consistente são necessários: amplo
% conhecimento dos fatores pertinentes, visão clara do problema e articulação lógica entre
% os diversos tipos de conhecimento utilizados.
% É aconselhável o uso de citações bibliográficas. Entretanto, devem seguir
% as mesmas normas dos demais trabalhos científicos. Assim, as citações diretas longas
% (com mais de três linhas) devem constituir um parágrafo independente, recuado a 4,0cm
% da margem esquerda, em letras corpo 10, com espaço simples entre linhas. Contudo, as
% citações diretas curtas devem ser inseridas no texto entre aspas. As citações indiretas
% também são inclusas no corpo do texto. 

A questão da previsão do preço do combustível é, assim como boa parte das questões de predição, 
um problema de regressão(\textit{regression})[ADICIONAR CITACAO]. Tais tipos de problema se caracterizam
[EMBAÇAMENTO]

Este trabalho faz uso de alguns dos mais amplamente utilizados algoritmos de regressão,
são eles \textit{Simple Linear Regression}, \textit{Ridge Regression}, \textit{Support Vector Regression} com \textit{kernel Linear}
e \textit{Radial basis function}.

Os metódos anteriormente citados foram testado usando a liguagem Python em conjunto com as bibliotecas open sources
\textit{Python Data Analysis Library}(pandas) e \textit{scikit-learn}, a primeira permite o tratamento de dados e visualização dos mesmos,
a segunda é uma biblioteca voltada para aprendizadode máquinas e que contém os modelos e outras ferramentas utilitárias como
\textit{cross validation}

Incialmente separou-se o \textit{dataset} utilizando a pandas, limitando-se aos já citados
dados que caracterizariam o problema, também foi adicionado uma nova coluna \textit{label} que
possuí o mesmo valor que o valor de revenda e é o \textit{target} deste trabalho.

Posteriormente, utilizando a biblioteca \textit{scikit-learn} o \textit{dataset} foi dividido em 
teste e treino na proporção 30\% para 70\%. Os dados então foram utilizados
paratreinar os modelos, performarem notas de acurácia e também testou-se as notas
de \textit{cross-validation} para garatir que não houver \textit{overfitting} nos modelos.